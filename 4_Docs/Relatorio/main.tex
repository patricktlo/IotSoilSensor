% ****************
% Template para o uso do estilo ufpr.cls, modificado de acordo com critérios do PPGEE/UFPR
%
% Gere o documento preferencialmente em formato PDF utlizando pdflatex
%
% Versão 1.0 - Beta - Prof. Carlos Marcelo Pedroso
%
% Antes de utilizar, instale o pacote abntex (apt-get install abntex)
%
%%%%%%%%%%%%%%%%%%%%%%%%%%%%%%%%%%%%
\documentclass[12pt,a4paper]{ufpr}
\usepackage[brazil]{babel}
% \usepackage[utf8]{inputenc}
% \usepackage{amssymb,amsfonts,amsmath}
% \usepackage{multirow}
% \usepackage{graphicx}
% \usepackage{subfigure}
% \usepackage{setspace}
% \usepackage{listings} 
% \usepackage{float}
% \usepackage{parskip}
% \usepackage{array,longtable}
% \usepackage{verbatim} 
% %\usepackage{breakurl}
% \usepackage[none]{hyphenat}
% \setlength{\emergencystretch}{5em}
% % \usepackage{abnt2cite}
% \usepackage{natbib}
%%%%%%%%%%%%%%%%%%%%%%%%%%%%%%%%%%%%
\usepackage{trivfloat}
\trivfloat{quadro}
\floatstyle{plaintop} % Forçar posição da legenda para o topo
\restylefloat{quadro} % Forçar posição da legenda para o topo
\renewcommand{\listquadroname}{Lista de Quadros} % Forçar texto na Lista de Quadros
\usepackage{listings} 
\usepackage{color} %red, green, blue, yellow, cyan, magenta, black, white
\definecolor{mygreen}{RGB}{28,172,0} % color values Red, Green, Blue
\definecolor{mylilas}{RGB}{170,55,241}
\newcommand{\highlight}[1]{\colorbox{yellow}{#1}} % destacar msg

%UTILIZADO PARA MATLAB
\lstset{%
    basicstyle=\fontsize{10}{9}\tt,
    language=Matlab,%
    alias tlmgr='/opt/texbin/tlmgr'%basicstyle=\color{red},
    breaklines=true,%
    morekeywords={matlab2tikz},
    keywordstyle=\color{blue},%
    morekeywords=[2]{1}, keywordstyle=[2]{\color{black}},
    identifierstyle=\color{black},%
    stringstyle=\color{mylilas},
    commentstyle=\color{mygreen},%
    extendedchars=true,
    inputencoding=latin1,
    showstringspaces=false,%without this there will be a symbol in the places where there is a space
    numbers=left,%
    numberstyle={\tiny \color{black}},% size of the numbers
    numbersep=8pt, % this defines how far the numbers are from the text
    emph=[1]{for,end,break},emphstyle=[1]\color{red}, %some words to emphasise
    %emph=[2]{word1,word2}, emphstyle=[2]{style},    
    literate=
        {Ç}{{\c{C}}}1
        {Ã}{{\~{A}}}1
        {á}{{\'{a}}}1
        {ENTÃO}{{\keywordstyle{ENT\~{A}O}}}5
        {SENÃO}{{\keywordstyle{SEN\~{A}O}}}5
        {FAÇA}{{\keywordstyle{FA\c{C}A}}}4,\
      }


% alterando o aspecto da cor azul
\definecolor{blue}{RGB}{41,5,195}

%%%%%%%%%%%%%%%%%%%%%%%%%%%%%%%%%%%%
%%% don't hyphenate %%%%%%%%%%%%%%%%
\tolerance=1
\emergencystretch=\maxdimen
\hyphenpenalty=10000
\hbadness=10000
%%%%%%%%%%%%%%%%%%%%%%%%%%%%%%%%%%%%
%\usepackage{algorithm}
%\usepackage{algorithmic}
%%%%%%%%%%%%%%%%%%%%%%%%%%%%%%%%%%%%
%% Utilizado para equações com parenteses maiores %%
\newcommand{\PR}[1]{\ensuremath{\left[#1\right]}}
\newcommand{\PC}[1]{\ensuremath{\left(#1\right)}}
\newcommand{\chav}[1]{\ensuremath{\left\{#1\right\}}}
%%%%%%%%%%%%%%%%%%%%%%%%%%%%%%%%%%%%%%%%%%%%%%%%%%%%
% \usepackage[alf,abnt-year-extra-label=yes,abnt-full-initials=no,abnt-emphasize=bf,abnt-and-type=e,abnt-repeated-author-omit=yes]{abntcite}
\usepackage{abntcite}
% Para trocar a fonte para um padrão parecido com o Arial, que é especificado pela ABNT
\usepackage{helvet}
\renewcommand{\familydefault}{\sfdefault}
\setcounter{secnumdepth}{3}    % n - numero de niveis de subsubsection numeradas
\setcounter{tocdepth}{3}       % coloca ate o nivel n no sumario
\parskip 0pt
\setlength{\parindent}{1.5cm}
% ****************************************************************************
\title{Desenvolvimento de uma solução IoT para medida remota de temperatura,
  umidade do ar e do solo com baixo custo e baixo consumo de energia}
\author{Patrick Thierry Lorusso El Omairi}
\advisortitle{Orientador}

\advisorname{Prof. Dr. Márlio José do Couto Bonfim}
\advisorplace{Departamento de Engenharia Elétrica, UFPR}
\concentrationarea{YYY}
\city{Curitiba}
\year{2018}
% \dedication{.}

% \banca        % nao insira o nome do orientador, ja eh feito automaticamente
% {Profa. Dra. Maria Maria Maria}{Instituto de Física, USP}
% {Prof. Dr. José Silva}{Departamento de Engenharia Elétrica, ITA}
% {Prof. Dr. Pedro Paulo}{Departamento de Engenharia Elétrica, UFPR} % se nao houver deixe em branco {}{}
% {}{}    % se houver um quarto membro na banca, inserir nome e instituicao

% \defesa{14 de abril de 2012} % dia em que foi realizada a defesa da dissertacao

\begin{document}
% Retire os comentários conforme a necessidade
\makecapadissertacao           % cria capa para dissertacao de mestrado %
\makerosto                     % cria folha de rosto para versao final da UFPR %
% \maketermo                     % ATENÇÃO: o termo de aprovação é fornecido pelo PPGEE após o cumprimento de todos os requisitos para obtenção do grau de mestre, e deve ser inserido no documento final.
\makededication                % elemento opcional: ver variavel dedication
%\singlespacing           % espacamento 1 - capa UFPR%
%\onehalfspacing          % espacamento 1/2 %
\doublespacing            % espacamento 2 - UFPR %
%\pagestyle{plain} 
%\pagestyle{headings}
%\pagenumbering{roman}
\pagestyle{empty}
\chapter*{Agradecimentos}

          % 
\singlespacing
%\chapter*{Resumo}
%\addcontentsline{toc}{chapter}{\MakeUppercase{Resumo}}
%Um dos elementos que se faz necessário para estabelecer uma comunicação de  Rádio Frequência (RF) é o receptor RF. Este além de demodular o sinal recebido, também deve realizar outros processos como amplificar, filtrar e converter o sinal que se encontra na frequência portadora para a banda base. Para que a menor potência possível seja dispensada respeitando o desempenho mínimo estipulado, faz-se essencial o estudo de seus blocos constituintes. O bloco estudado até o momento é o Amplificador de Baixo Ruído (LNA). O LNA é o primeiro bloco ativo encontrado no receptor, deve amplificar o sinal de entrada para o misturador.

Atentamos em desenvolver simulações do bloco LNA que faz parte do bloco de Rádio Frequência relacionando parâmetros de desempenho como ganho (G) , figura de ruído (NF) e ponto de interceptação de terceira ordem (IIP3)  com a energia gasta pelo componente, para poder identificar qual seria a melhor composição geral de um sistema de recepção em termos de consumo. Após se familiarizar com as arquiteturas de recepção RF empregamos a leitura de artigos científicos do elemento LNA, procura-se encontrar por fim, uma relação que melhor se adeque ao consumo do bloco com os parâmetros de desempenho sistêmico (G, NF, IIP3, e frequência de operação). 

\hfill \break

\noindent \textbf{Palavras Chave} Rádio Frequência, Amplificador de Baixo Ruído, LNA, Ruído, Ganho, IIP3.
%\chapter*{Abstract}
%\addcontentsline{toc}{chapter}{\MakeUppercase{Abstract}}
%Wireless communication services demand an increasing volume of transmitted data while the portability imposes an increasing autonomy of the communication devices, such that an efficient information transmission is mandatory. In addiction, with the increase in the wireless communication demand, and the continuous shortage of free passband, it is stimulated to develop networks, transmission and reception smart systems, capable of watching, learning and adapting themselves to the transmission environment around them.

This project aims to come up with solutions in base-band and as transmission and reception systems for radio-frequency with the goal of increasing the energetic efficiency in the wireless data transmission as well as creating a system of optimized and opportunistic use of the communication channel. From the system point of view, the main idea is to understand flexible RF transmitters and receptors architectures capable of adapting themselves to different communication policies as well as having configurable performance. This characteristic allows the minimum energy usage in function of the communication channel quality that is presented in a specific moment.

In order to achieve that, cognitive radio techniques and software controlled radio techniques will be explored and applied in a simulation platform that shall take into consideration the global usage of a wireless communication network.

Initially, it will be done a Simulink/Matlab analysis of the data and equations for, later, building a functional prototype. 

\hfill \break

\noindent \textbf{Key-words:} Cognitive. Radio. Matlab. Frequency. Spectre. Radiofrequency. Simulink. Simulation.

\newpage

\listoffigures
%\listoftables
\newpage

\chapter*{Lista de Siglas}
\begin{tabular}{p{3cm}p{15cm}}

IoT & Internet das coisas\\


  
\end{tabular} 

\newpage
\tableofcontents
\newpage
\pagenumbering{arabic} % numerais romanos
\pagestyle{headings}   % números de pagina no cabeçalho
\setcounter{page}{9}  % substitua pelo número inicial da primeira página do capítulo 1
% Texto deve ter espaçamento duplo
\doublespacing

%**********************************************************

 \chapter{Introdução} \label{chapter:intro}

\vspace{-0.1cm}

\section{Problematização}

\par
Com a crescente demanda para uma produção agrícola mais eficiente, devido ao
crescimento populacional em escala mundial e nacional, existe uma demanda muito grande
para métodos mais eficientes e ecológicos de agricultura.

Uma das maneiras de deixar mais eficiente o plantio, é medindo grandezas no
campo, como por exemplo a temperatura, a umidade do solo e do ar. O
acompanhamento dessas grandezas em tempo real pode dar informações essenciais
para o fazendeiro. Por exemplo, tendo informações sobre a umidade do solo, será
possível saber quando será o momento ideal para irrigar os campos.

Com vários sensores capazes de medir tais grandezas espalhados pela fazenda, o
fazendeiro seria capaz de obter informações mais específicas sobre determinadas
partes do campo, como a distribuição da temperatura, da umidade do solo e
relativa do ar, e derivados destes.

Este projeto visa desenvolver um sensor capaz de medir a
temperatura, a umidade relativa do ar e a umidade do solo, capaz de
enviar essas informações por uma rede \textit{IoT}, mais especificamente,
\textit{LoRa}, tendo um consumo extremamente baixo e consequentemente uma
grande autonomia.

\section{Objetivos}

\subsection{Objetivo Geral}
Desenvolver um dispositivo eletrônico capaz de medir temperatura, umidade relativa do ar
e do solo, e enviar os dados por uma rede \textit{IoT LoRa}. Todos os sensores
integrados junto do microcontrolador numa só placa de baixo custo e baixo consumo de energia.
O sensor de umidade do solo será capacitivo para que ele tenha uma longa duração
e será integrado dentro da placa de circuito impresso.

\subsection{Objetivos Específicos}
\begin{itemize}
  \itemsep0em\item Desenvolvimento do hardware.
  \item Simulação do hardware no LTSpice ou em \textit{scripts} Python
  \item Desenvolvimento de uma placa de circuito impresso
  \item Desenvolvimento do software embarcado para microcontrolador STM32

\end{itemize}

%\subsection{Normas e padrões utilizados}

%\begin{itemize}
%\itemsep0em\item Norma ABNT.
%\end{itemize}

\section{Justificativa}

As exportações Brasileiras ligadas ao agronegócio chegam a 41,5\% das vendas
externas totais no país, onde somente a soja é o principal produto exportado com
42,5\% das exportações nacionais do agronegócio \cite{fiesp2018}.

Além disso, a produção de carnes, outro setor muito importante para a exportação
nacional, depende diretamente da agricultura uma vez que as rações dos animais
são feitas com produtos provenientes da agricultura.

Sendo então a agricultura uma área essencial para o desenvolvimento
econômico e sustentável do Brasil, inovações nessa área tem um potencial enorme
para deixar mais eficiente o crescimento das plantas, diminuindo gastos com
água, por exemplo.

Um dispositivo eletrônico capaz de disponibilizar essas informações do campo
para o fazendeiro pode ser essencial para o desenvolvimento do negócio, uma vez
que o fazendeiro poderá monitorar seu plantio. E com o surgimento das redes
\textit{IoT}, se torna mais viável colocar vários dispositivos desses, de baixo custo,
espalhados pelo campo para conseguir o maior número de dados possível.


 
% \vspace{-0.3cm}

\chapter{FUNDAMENTAÇÃO TEÓRICA} \label{chapter:fund}


% \chapter{Metodologia} \label{chapter:metodologia}

% \chapter{Projeto} \label{chapter:projeto}

% \input{capitulo5.tex}
% \input{capitulo6.tex}
% \vspace{-0.3cm}


\chapter{Conclusão} \label{chapter:conclusao}


% \chapter{Cronograma} \label{chapter:cronograma}


\section{Etapas}

\newpage

\section{Cronograma Semestral das Etapas de Desenvolvimento de Trabalho}


% Please add the following required packages to your document preamble:
% \usepackage{multirow}


%**********************************************************

\appendix
% \renewcommand{\thesubsection}{A}

\singlespacing
%\chapter{Referências Artigos para Coleta de Dados} 





\doublespacing
\parskip 10pt
% \bibliographystyle{sbc}      	% O formato da SBC me parece mais agradável e é bastante semelhante ao exigido pela ABNT 
% \bibliographystyle{abnt-num}	% Referências alfa numéricas, no estilo [1] [2] ...
\bibliographystyle{abnt-alf}    % Formato desenvolvido para atender requisitos da ABNT (Open Source), do projeto ABNTEX (http://codigolivre.org.br)
\bibliography{referencias}
\addcontentsline{toc}{chapter}{\MakeUppercase{Bibliografia}}
%\singlespacing
%\makecapadissertacao
\end{document}