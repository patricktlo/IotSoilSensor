\chapter{Introdução} \label{chapter:intro}

\vspace{-0.1cm}

\section{Problematização}

\par
Com a crescente demanda para uma produção agrícola mais eficiente, devido ao
crescimento populacional em escala mundial e nacional, existe uma demanda muito grande
para métodos mais eficientes e ecológicos de agricultura.

Uma das maneiras de deixar mais eficiente o plantio, é medindo grandezas no
campo, como por exemplo a temperatura, a umidade do solo e do ar. O
acompanhamento dessas grandezas em tempo real pode dar informações essenciais
para o fazendeiro. Por exemplo, tendo informações sobre a umidade do solo, será
possível saber quando será o momento ideal para irrigar os campos.

Com vários sensores capazes de medir tais grandezas espalhados pela fazenda, o
fazendeiro seria capaz de obter informações mais específicas sobre determinadas
partes do campo, como a distribuição da temperatura, da umidade do solo e
relativa do ar, e derivados destes.

Este projeto visa desenvolver um sensor capaz de medir a
temperatura, a umidade relativa do ar e a umidade do solo, capaz de
enviar essas informações por uma rede \textit{IoT}, mais especificamente,
\textit{LoRa}, tendo um consumo extremamente baixo e consequentemente uma
grande autonomia.

\section{Objetivos}

\subsection{Objetivo Geral}
Desenvolver um dispositivo eletrônico capaz de medir temperatura, umidade relativa do ar
e do solo, e enviar os dados por uma rede \textit{IoT LoRa}. Todos os sensores
integrados junto do microcontrolador numa só placa de baixo custo e baixo consumo de energia.
O sensor de umidade do solo será capacitivo para que ele tenha uma longa duração
e será integrado dentro da placa de circuito impresso.

\subsection{Objetivos Específicos}
\begin{itemize}
  \itemsep0em\item Desenvolvimento do hardware.
  \item Simulação do hardware no LTSpice ou em \textit{scripts} Python
  \item Desenvolvimento de uma placa de circuito impresso
  \item Desenvolvimento do software embarcado para microcontrolador STM32

\end{itemize}

%\subsection{Normas e padrões utilizados}

%\begin{itemize}
%\itemsep0em\item Norma ABNT.
%\end{itemize}

\section{Justificativa}

As exportações Brasileiras ligadas ao agronegócio chegam a 41,5\% das vendas
externas totais no país, onde somente a soja é o principal produto exportado com
42,5\% das exportações nacionais do agronegócio \cite{fiesp2018}.

Além disso, a produção de carnes, outro setor muito importante para a exportação
nacional, depende diretamente da agricultura uma vez que as rações dos animais
são feitas com produtos provenientes da agricultura.

Sendo então a agricultura uma área essencial para o desenvolvimento
econômico e sustentável do Brasil, inovações nessa área tem um potencial enorme
para deixar mais eficiente o crescimento das plantas, diminuindo gastos com
água, por exemplo.

Um dispositivo eletrônico capaz de disponibilizar essas informações do campo
para o fazendeiro pode ser essencial para o desenvolvimento do negócio, uma vez
que o fazendeiro poderá monitorar seu plantio. E com o surgimento das redes
\textit{IoT}, se torna mais viável colocar vários dispositivos desses, de baixo custo,
espalhados pelo campo para conseguir o maior número de dados possível.

