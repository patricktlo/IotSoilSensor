\chapter{Introdução} \label{chapter:intro}

\par 

\vspace{-0.1cm}

\section{Objetivos}

\subsection{Objetivo Geral}
Desenvolver um dispositivo eletrônico capaz de medir temperatura, umidade do ar
e do solo, e enviar os dados por uma rede \textit{IoT LoRa}. Todos os sensores integrados numa placa
de baixo custo e baixo consumo de energia.
O sensor de umidade do solo será capacitivo para que ele tenha uma longa duração e será integrado
dentro da placa de circuito impresso.
Usando a rede LoRa, seria teoricamente possível de ligar várias placas na rede facilmente, sendo
possível de monitorar um grande espaço com vários sensores.
Além disso será necessário dimensionar a bateria (que deverá durar por meses ou até anos) e a antena
para a rede LoRa.

\subsection{Objetivos Específicos}
\begin{itemize}
\itemsep0em\item 

\end{itemize}

\subsection{Normas e padrões utilizados}

\begin{itemize}
\itemsep0em\item Norma ABNT.

\end{itemize}

