Um dos elementos que se faz necessário para estabelecer uma comunicação de  Rádio Frequência (RF) é o receptor RF. Este além de demodular o sinal recebido, também deve realizar outros processos como amplificar, filtrar e converter o sinal que se encontra na frequência portadora para a banda base. Para que a menor potência possível seja dispensada respeitando o desempenho mínimo estipulado, faz-se essencial o estudo de seus blocos constituintes. O bloco estudado até o momento é o Amplificador de Baixo Ruído (LNA). O LNA é o primeiro bloco ativo encontrado no receptor, deve amplificar o sinal de entrada para o misturador.

Atentamos em desenvolver simulações do bloco LNA que faz parte do bloco de Rádio Frequência relacionando parâmetros de desempenho como ganho (G) , figura de ruído (NF) e ponto de interceptação de terceira ordem (IIP3)  com a energia gasta pelo componente, para poder identificar qual seria a melhor composição geral de um sistema de recepção em termos de consumo. Após se familiarizar com as arquiteturas de recepção RF empregamos a leitura de artigos científicos do elemento LNA, procura-se encontrar por fim, uma relação que melhor se adeque ao consumo do bloco com os parâmetros de desempenho sistêmico (G, NF, IIP3, e frequência de operação). 

\hfill \break

\noindent \textbf{Palavras Chave} Rádio Frequência, Amplificador de Baixo Ruído, LNA, Ruído, Ganho, IIP3.