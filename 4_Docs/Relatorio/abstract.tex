Wireless communication services demand an increasing volume of transmitted data while the portability imposes an increasing autonomy of the communication devices, such that an efficient information transmission is mandatory. In addiction, with the increase in the wireless communication demand, and the continuous shortage of free passband, it is stimulated to develop networks, transmission and reception smart systems, capable of watching, learning and adapting themselves to the transmission environment around them.

This project aims to come up with solutions in base-band and as transmission and reception systems for radio-frequency with the goal of increasing the energetic efficiency in the wireless data transmission as well as creating a system of optimized and opportunistic use of the communication channel. From the system point of view, the main idea is to understand flexible RF transmitters and receptors architectures capable of adapting themselves to different communication policies as well as having configurable performance. This characteristic allows the minimum energy usage in function of the communication channel quality that is presented in a specific moment.

In order to achieve that, cognitive radio techniques and software controlled radio techniques will be explored and applied in a simulation platform that shall take into consideration the global usage of a wireless communication network.

Initially, it will be done a Simulink/Matlab analysis of the data and equations for, later, building a functional prototype. 

\hfill \break

\noindent \textbf{Key-words:} Cognitive. Radio. Matlab. Frequency. Spectre. Radiofrequency. Simulink. Simulation.