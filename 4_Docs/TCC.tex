\documentclass[
	12pt,				% tamanho da fonte
	openright,			% capítulos começam em pág ímpar (insere página vazia caso preciso)
	oneside,			% para impressão apenas em um lado do papel
	a4paper,			% tamanho do papel.
	brazil				% o último idioma é o principal do documento
	]{abntex2}

\usepackage{etoolbox}
\usepackage{lmodern}			% Usa a fonte Latin Modern			
\usepackage[T1]{fontenc}		% Selecao de codigos de fonte.
\usepackage[utf8]{inputenc}		% Codificacao do documento (conversão automática dos acentos)
%\usepackage{lastpage}			% Usado pela Ficha catalográfica
\usepackage{indentfirst}		% Indenta o primeiro parágrafo de cada seção.
\setlength{\parindent}{1.5cm}   % Espaçamento de 1,5cm do parágrafo
\usepackage{color}				% Controle das cores
\usepackage{graphicx}			% Inclusão de gráficos
\usepackage{microtype} 			% para melhorias de justificação
\usepackage{lipsum}				% para geração de dummy text
\usepackage[alf]{abntex2cite}	% Citações padrão ABNT
\usepackage[table,xcdraw]{xcolor}% Cédula colorida em tabelas
%\usepackage{pdflscape}          % Rotaciona página
\usepackage{Capa}               % Capa e folha de rosto com modificações
\usepackage{float}              % Melhor posicionamento de figuras
\usepackage{gensymb}            % símbolo º
\usepackage[justification=justified,singlelinecheck=false]{caption}
%\usepackage{etoolbox}           % Configurações adicionais de macros
\usepackage{xparse}						
\usepackage{multirow}
\NewDocumentCommand\cc{+u{\cc}}{\ignorespaces}

% --------------------
% Dados do Documento
% --------------------

\titulo{Título do TCC }
\autor{Nome Aluno 1  \\  Nome Aluno 2}
\data{2016}
\instituicao{Universidade Federal do Paraná
%             \par
%             Setor de Ciências Exatas
%             \par
%             Departamento de Estatística
%             \par
%             Curso de Estatística
            }
\local{Curitiba}
\orientador[Orientadora:]{Profa. Dra. Suely Ruiz Giolo}
\preambulo{Trabalho de Conclusão de Curso apresentado à disciplina Laboratório B do Curso de Graduação em Estatística da Universidade Federal do Paraná,
como exigência parcial para obtenção do grau de Bacharel em Estatística.}

% ---------------------
% Configurações básicas
% --------------------

% informações do PDF
\makeatletter

\hypersetup{
     	%pagebackref=true,
		pdftitle={\@title},
		pdfauthor={\@author},
    	pdfsubject={\imprimirpreambulo},
        pdfkeywords = {Desempenho Acadêmico}{Estatística Descritiva}{Evasão}{Vestibular},
		colorlinks=true,       		% false: boxed links; true: colored links
    	linkcolor=black,          	% color of internal links
    	citecolor=black,      	    % color of links to bibliography
    	filecolor=magenta,      	% color of file links
		urlcolor=black,
		bookmarksdepth=4
}
\makeatother

\graphicspath{{Figuras/}}

% -------------------
% Início do documento
% -------------------

\begin{document}

\frenchspacing % Retira espaço extra obsoleto entre as frases.

% ----------------------------------------------------------
% ELEMENTOS PRÉ-TEXTUAIS
% ----------------------------------------------------------

% ----
% Capa
% ----
\imprimircapa
% ---

% --------------
% Folha de rosto
% --------------
\imprimirfolhaderosto
% ---

%\begin{dedicatoria}  % Opcional
%\vspace*{\fill}
% Dedico este trabalho aos meus pais .....
%\vspace*{\fill}
%\end{dedicatoria}

% --------------
% Agradecimentos  % Opcional
% --------------

\begin{agradecimentos}

Inclua seus agradecimentos aqui. Há vários exemplos na internet

http://www.testeievoce.com.br/2012/09/agradecimento-do-meu-tcc.html

http://www.tccmonografiaseartigos.com.br/agradecimentos-tcc-monografia-trabalho
\end{agradecimentos}

\begin{epigrafe} % Opcional
\vspace*{\fill}
\begin{flushright}
\textit{"Democracia é oportunizar a todos o mesmo ponto de partida. \\
         Quanto ao ponto de chegada, depende de cada um".\\
         (Fernando Sabino)}
\end{flushright}
\end{epigrafe}

% ------
% Resumo
% ------
\newpage
\setlength{\absparsep}{18pt}   % ajusta o espaçamento dos parágrafos do resumo
\setlength{\abstitleskip}{1cm} % adiciona mais um cm após o 'titulo' do Resumo para ficar com 2cm,

\begin{resumo}

Digite aqui o resumo do trabalho. Deve conter de 150 a 500 palavras.

\textbf{Palavras-chave}: Estatística Descritiva. Regressão. Resíduos.
\end{resumo}

% -------------------------------
% Lista de abreviaturas e siglas - Opcional
% -------------------------------

%\begin{siglas}  %
%  \item[IES]  Instituição de Ensino Superior
%  \item[UFG]  Universidade Federal de Goiás
%  \item[UFPR] Universidade Federal do Paraná
%\end{siglas}


% -------
% Sumario
% -------
\pdfbookmark[0]{\contentsname}{toc}
\tableofcontents*
\cleardoublepage
% ---

\makepagestyle{abntheadings}
\makeevenhead{abntheadings}{\ABNTEXfontereduzida\thepage}{}{}
\makeoddhead{abntheadings}{}{}{\ABNTEXfontereduzida\thepage}
\makeheadrule{abntheadings}{\textwidth}{0in}

% ----------------------------------------------------------
% ELEMENTOS TEXTUAIS
% ----------------------------------------------------------

\textual

\chapter{Introdução}
\input{./Modulos/Introducao}

\chapter{Revisão de Literatura}
\input{./Modulos/Revisao}

\chapter{Material e Métodos}
\input{./Modulos/MaterialeMetodos}

\chapter{Resultados e Discussão}
\input{./Modulos/Resultados}

\chapter{Considerações Finais}
\input{./Modulos/ConsFinais}

% ----------------------------------------------------------
% Referências bibliográficas
% ----------------------------------------------------------

\setlength{\afterchapskip}{\baselineskip}

\bibliography{Referencias}

% ----------------------------------------------------------
% ELEMENTOS PÓS-TEXTUAIS
% ----------------------------------------------------------
\postextual
% ----------------------------------------------------------

% ----------------------------------------------------------
% ANEXOS
% ----------------------------------------------------------

\begin{apendicesenv}

\partapendices

\chapter*{\normalsize APÊNDICE A - Digite o cabeçalho do apêndice}

Apêndice: texto ou documento elaborado pelo autor, a fim de complementar sua argumentação, sem prejuízo da
unidade nuclear do trabalho.

\chapter*{\normalsize APÊNDICE B - Digite o cabeçalho do apêndice}

\end{apendicesenv}

\begin{anexosenv}

\partanexos
%\addcontentsline{toc}{chapter}{\hspace{2.105cm}ANEXOS}
\renewcommand{\ABNTEXchapterfontsize}{\ABNTEXsectionfont}

\chapter*{\normalsize ANEXO A - Digite o cabeçalho do anexo}

Anexo: texto ou documento não elaborado pelo autor, que serve de fundamentação, comprovação e ilustração

\chapter*{\normalsize ANEXO B - Digite o cabeçalho do anexo}

\end{anexosenv}


\end{document}
